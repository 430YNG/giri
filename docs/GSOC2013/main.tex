\documentclass[DIV=calc, paper=a4, fontsize=11pt, twocolumn]{scrartcl}

\usepackage{lipsum} % Used for inserting dummy 'Lorem ipsum' text into the template
\usepackage[english]{babel} % English language/hyphenation
\usepackage[protrusion=true,expansion=true]{microtype} % Better typography
\usepackage{amsmath,amsfonts,amsthm} % Math packages
\usepackage[svgnames]{xcolor} % Enabling colors by their 'svgnames'
\usepackage[hang, small,labelfont=bf,up,textfont=it,up]{caption} % Custom captions under/above floats in tables or figures
\usepackage{booktabs} % Horizontal rules in tables
\usepackage{fix-cm}	 % Custom font sizes - used for the initial letter in the document

\usepackage{sectsty} % Enables custom section titles
\allsectionsfont{\usefont{OT1}{phv}{b}{n}} % Change the font of all section commands

\usepackage{fancyhdr} % Needed to define custom headers/footers
\pagestyle{fancy} % Enables the custom headers/footers
\usepackage{lastpage} % Used to determine the number of pages in the document (for "Page X of Total")
\usepackage[colorlinks=false]{hyperref}
\usepackage{paralist}

% Headers - all currently empty
\lhead{}
\chead{}
\rhead{}

% Footers
\lfoot{}
\cfoot{}
\rfoot{\footnotesize Page \thepage\ of \pageref{LastPage}} % "Page 1 of 2"

\renewcommand{\headrulewidth}{0.0pt} % No header rule
\renewcommand{\footrulewidth}{0.4pt} % Thin footer rule

\usepackage{lettrine} % Package to accentuate the first letter of the text
\newcommand{\initial}[1]{ % Defines the command and style for the first letter
\lettrine[lines=3,lhang=0.3,nindent=0em]{
\color{DarkGoldenrod}
{\textsf{#1}}}{}}

%----------------------------------------------------------------------------------------
%	TITLE SECTION
%----------------------------------------------------------------------------------------

\usepackage{titling} % Allows custom title configuration

\newcommand{\HorRule}{\color{DarkGoldenrod} \rule{\linewidth}{1pt}} % Defines the gold horizontal rule around the title

\begin{document}

\pretitle{\vspace{-50pt} \begin{flushleft} \HorRule \fontsize{40}{40} \usefont{OT1}{phv}{b}{n} \color{DarkRed} \selectfont} % Horizontal rule before the title

\title{{\large A GSoC 2013 Final Report}\\Enhancing Giri: \\Dynamic Slicing in LLVM}

\posttitle{\par\end{flushleft}\vskip 0.5em} % Whitespace under the title
\preauthor{\begin{flushleft}\large \lineskip 0.5em \usefont{OT1}{phv}{b}{sl} \color{DarkRed}} % Author font configuration

\author{Mingliang Liu, Swarup Kumar Sahoo} % Your name

\postauthor{\footnotesize \usefont{OT1}{phv}{m}{sl} \color{Black} % Configuration for the institution name
\\http://github.com/liuml07/giri% Your institution

\par\end{flushleft}\HorRule} % Horizontal rule after the title

\date{} % Add a date here if you would like one to appear underneath the title block
%----------------------------------------------------------------------------------------

\maketitle % Print the title

\thispagestyle{fancy} % Enabling the custom headers/footers for the first page 

%----------------------------------------------------------------------------------------
%	ABSTRACT
%----------------------------------------------------------------------------------------

% The first character should be within \initial{}
\initial{D}\textbf{ynamic program slicing has been used in many applications.
Giri was a research project from UIUC, which implemented the dynamic backward slicing in LLVM.
It was selected as the Google Summer of Code 2013.
We achieved several improvements to Giri during GSoC 2013:
\begin{inparaenum}[\itshape 1\upshape)]
\item Update the code to LLVM mainline and make it robust,
\item Reduce the trace size,
\item Make the Giri thread-aware (pthread only),
\item Improve the performance of run-time.
\end{inparaenum}
}

%------------------------------------------------
\section{Introduction}
Dynamic program slice contains statements that influence the value of a variable/instruction occurrence for given program inputs~\cite{agrawal1990dynamic}.
The traditional program slicing was called static program slicing, which was firstly proposed by Weiser~\cite{weiser}.
There are many applications that use (or could benefit from) dynamic slicing, both by research and industry organizations (e.g. Microsoft, IBM).
For example, it's long been used in software debugging~\cite{1993debugging,1999efficient} and testing~\cite{1993incremental}.
However, as far as we're concerned, there is no publicly available dynamic slicing tool in either GCC or Open64.

Sahoo et. al. from UIUC use dynamic program slicing to generate likely invariants for automated software fault localization~\cite{sahoo2013asplos}.
They implemented the dynamic program slicing code, called Giri, in LLVM compiler infrastructure (version 2.6) for research purpose.
It traces the user program execution and reports the dynamic slices in the end.
It also maps LLVM IR statements to source code for its output using the debug metadata.
To make the Giri up to date and robust for general usage,
we enhanced its code in several factors.

The report is organized as following.
Section~\ref{sec:overview} shows the internal architecture of Giri, including how it works and how to use it.
Section~\ref{sec:progress} lists the progress we made during the GSoC 2013.
Section~\ref{sec:todo} is the TODO list of Giri as future work.
Section~\ref{sec:contact} concludes our work, and attaches our project URL.

%------------------------------------------------
\section{Overview of Giri}
\label{sec:overview}

%------------------------------------------------
\section{Progress}
\label{sec:progress}
Our goal of GSoC was to make the Giri code update to the latest LLVM version, improve its runtime performance, and/or reduce the tracing cost.
There are several things we did in this summer:
\begin{enumerate}
	\item \textbf{Update the code to LLVM mainline.}
		\begin{itemize}
			\item Release a v3.1 version which works with LLVM 3.1
			\item Make the code update to the latest LLVM 3.4 (\texttt{r191529})
		\end{itemize}
	\item \textbf{Make the Giri run-time library thread safe and the slicing pass thread aware}.
		The operation traces of all threads are written to a single file.
		Trace records include thread id (\texttt{pthread\_t}), which indicates the thread performing a particular operation.
		We use locks when write traces to file in order to avoid the race condition.
	\item \textbf{Improve the Giri run-time performance.}
		We fixed the bug of \texttt{mmap} function call at the end of the tracing.
		Giri now dynamically computes the cache size of trace records to hold in memory before flushing to disk.
	\item \textbf{Write dozens of unit tests and try several real programs.}
		There is also a simple test framework which runs all the unit tests at the top level of \texttt{giri/test/} and report the result.
		In the future development, every patch should be checked before committing it to the git repository.
	\item \textbf{Reduce the trace size.}
		We truncate the file at the end of tracer according to its real size.
	\item \textbf{Write documents for the project.}\\
		There are several sample pages:
		\begin{itemize}
			\item \href{https://github.com/liuml07/giri/wiki}{The Wiki Page Home}.
			\item \href{https://github.com/liuml07/giri/wiki/How-to-Compile-Giri}{How to Compile Giri}.
			\item \href{https://github.com/liuml07/giri/wiki/Hello-World}{Hello World!}
			\item \href{https://github.com/liuml07/giri/wiki/Example-Usage}{Example Usage With An Example}.
		\end{itemize}
\end{enumerate}

%------------------------------------------------
\section{TODO List}
The following are the TODO list we're going to address in the next few months.
An update one is maintained online at the \href{https://github.com/liuml07/giri/wiki/TODO}{wiki page of Giri at github.com}.
\label{sec:todo}
\begin{enumerate}
	\item Improve the performance of locking mechanism at the runtime
	\item Write more unit tests with complicated direct/indirect recursive function calls
	\item Double check the \texttt{TraceFile.cpp} to make sure the thread id be checked correctly
	\item Add \texttt{tool/Tracer.cpp} to the make list. It was removed when upgrading to LLVM 3.4
	\item Pass more real world programs and add them to the \texttt{test/} directory
	\item Try large test programs, e.g. \emph{nginx}, \emph{squid}, etc
	\item Make Giri code runnable at other platforms besides Linux, e.g. Mac OS X, Cygwin, and FreeBSD.
	\item Consider more special function calls in \texttt{TracingNoGiri::visitSpecialCall()} function of the tracing pass
	\item Parallelize the code writing the entry cache to trace file and adding entry to the cache.
\end{enumerate}

%------------------------------------------------
\section{Conclusion}
\label{sec:contact}
As the first publicly available dynamic slicing tool,
Giri was made better during the GSoC. We updated the code to LLVM mainline, reduced the trace size, made it thread-aware, and improve the performance of its run-time.
We publish our code at \href{https://github.com/liuml07/giri}{https://github.com/liuml07/giri}~\cite{giri}.

The code is still under active development.
Dr. Swarup will direct Mingliang LIU in the future to make the Giri code better.
There are other guys from Tsinghua University, Xi'an Jiaotong University and University of Miami,
who wrote to us in email hoping to use and contribute to Giri project.
For more information, please visit the homepage above.

%------------------------------------------------
\bibliographystyle{plain}
\bibliography{references} 

\end{document}
