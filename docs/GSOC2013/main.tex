\documentclass[DIV=calc, paper=a4, fontsize=11pt, twocolumn]{scrartcl}

\usepackage{lipsum} % Used for inserting dummy 'Lorem ipsum' text into the template
\usepackage[english]{babel} % English language/hyphenation
\usepackage[protrusion=true,expansion=true]{microtype} % Better typography
\usepackage{amsmath,amsfonts,amsthm} % Math packages
\usepackage[svgnames]{xcolor} % Enabling colors by their 'svgnames'
\usepackage[hang, small,labelfont=bf,up,textfont=it,up]{caption} % Custom captions under/above floats in tables or figures
\usepackage{booktabs} % Horizontal rules in tables
\usepackage{fix-cm}	 % Custom font sizes - used for the initial letter in the document

\usepackage{sectsty} % Enables custom section titles
\allsectionsfont{\usefont{OT1}{phv}{b}{n}} % Change the font of all section commands

\usepackage{fancyhdr} % Needed to define custom headers/footers
\pagestyle{fancy} % Enables the custom headers/footers
\usepackage{lastpage} % Used to determine the number of pages in the document (for "Page X of Total")
\usepackage{paralist}
\usepackage[colorlinks=false]{hyperref}
\usepackage{listings}
\usepackage{color}

\lstdefinestyle{giricode}{
	belowcaptionskip=1\baselineskip,
	breaklines=false,
	frame=tb,
	numbers=left,                    % where to put the line-numbers; possible values are (none, left, right)
	numbersep=5pt,                   % how far the line-numbers are from the code
	numberstyle=\tiny\color{gray}, % the style that is used for the line-numbers
	xleftmargin=\parindent,
	language=C,
	showstringspaces=false,
	basicstyle=\footnotesize\ttfamily,
	keywordstyle=\bfseries\color{green!40!black},
	commentstyle=\itshape\color{purple!40!black},
	identifierstyle=\color{blue},
	stringstyle=\color{orange},
}

% Headers - all currently empty
\lhead{}
\chead{}
\rhead{}

% Footers
\lfoot{}
\cfoot{}
\rfoot{\footnotesize Page \thepage\ of \pageref{LastPage}} % "Page 1 of 2"

\renewcommand{\headrulewidth}{0.0pt} % No header rule
\renewcommand{\footrulewidth}{0.4pt} % Thin footer rule

\usepackage{lettrine} % Package to accentuate the first letter of the text
\newcommand{\initial}[1]{ % Defines the command and style for the first letter
\lettrine[lines=3,lhang=0.3,nindent=0em]{
\color{DarkGoldenrod}
{\textsf{#1}}}{}}

%----------------------------------------------------------------------------------------
%	TITLE SECTION
%----------------------------------------------------------------------------------------

\usepackage{titling} % Allows custom title configuration

\newcommand{\HorRule}{\color{DarkGoldenrod} \rule{\linewidth}{1pt}} % Defines the gold horizontal rule around the title

\begin{document}

\pretitle{\vspace{-50pt} \begin{flushleft} \HorRule \fontsize{40}{40} \usefont{OT1}{phv}{b}{n} \color{DarkRed} \selectfont} % Horizontal rule before the title

\title{{\large A GSoC 2013 Final Report}\\Enhancing Giri: \\Dynamic Slicing in LLVM}

\posttitle{\par\end{flushleft}\vskip 0.5em} % Whitespace under the title
\preauthor{\begin{flushleft}\large \lineskip 0.5em \usefont{OT1}{phv}{b}{sl} \color{DarkRed}} % Author font configuration

\author{Mingliang Liu, Swarup Kumar Sahoo} % Your name

\postauthor{\footnotesize \usefont{OT1}{phv}{m}{sl} \color{Black} % Configuration for the institution name
\\http://github.com/liuml07/giri% Your institution

\par\end{flushleft}\HorRule} % Horizontal rule after the title

\date{} % Add a date here if you would like one to appear underneath the title block
%----------------------------------------------------------------------------------------

\maketitle % Print the title

\thispagestyle{fancy} % Enabling the custom headers/footers for the first page 

%----------------------------------------------------------------------------------------
%	ABSTRACT
%----------------------------------------------------------------------------------------

% The first character should be within \initial{}
\initial{D}\textbf{ynamic program slicing has been used in many applications.
Giri was a research project from UIUC, which implemented the dynamic backward slicing in LLVM.
It was selected as the Google Summer of Code 2013.
We achieved several improvements to Giri during GSoC 2013:
\begin{inparaenum}[\itshape 1\upshape)]
\item Update the code to LLVM mainline and make it robust,
\item Reduce the trace size,
\item Make the Giri thread-aware (pthread only),
\item Improve the performance of run-time.
\end{inparaenum}
}

%------------------------------------------------
\section{Introduction}
Program slice contains all statements in a program that directly or indirectly affect the value of a variable or instruction~\cite{weiser}.
We can furthure narrow down the notion of the \emph{slice},
which contains dynamic program statements that influence the instance of a variable or instruction for \textbf{given} program inputs.
This is called \emph{dynamic program slicing}~\cite{agrawal1990dynamic}.
There are many applications from research and industry organizations which use (or could benefit from) dynamic slicing.
For example, it has long been used in software debugging~\cite{1993debugging,1999efficient} and testing~\cite{1993incremental}.
However, to the best of our knowledge, there is no publicly available dynamic slicing tool in either GCC or Open64.

Sahoo et. al. use dynamic program slicing to filter/remove false positive candidate root causes for automated software fault localization~\cite{sahoo2013asplos}.
They implemented the dynamic program slicing code, named Giri, in LLVM compiler infrastructure (version 2.6) for research purpose.
It collects the trace of a user program execution and finally reports the dynamic slice of a given statement.
It also maps LLVM IR statements to source code using the debug metadata to report the dynamic slice in terms of source line numbers.
To make the Giri up-to-date and robust for general use, we enhanced the code in several ways.

The report is organized as following.
Section~\ref{sec:overview} shows the internal architecture of Giri, including how it works and how to use it.
Section~\ref{sec:progress} lists the progress we made during the GSoC 2013.
Section~\ref{sec:todo} lists the future work to improve Giri further, instead of TODO (everywhere) write future work.
Section~\ref{sec:contact} concludes our work.

%------------------------------------------------
\section{Overview}
\subsection{Dynamic Program Slicing}
\label{sec:overview}
The following is an example program.
\lstinputlisting[language=C, style=giricode]{../../test/UnitTests/test14/struct-ptr.c}
Given the input \texttt{x = 10}, the source line numbers affect the return value are: $<$\emph{27, 40, 42, 44, 47}$>$.

The default start point of slicing is the \texttt{return} instruction at the \texttt{main} function.
Note that there is only one \texttt{return} when the main executes.
Giri also supports two more ways to specify the slicing criterion, which are source code line number and LLVM instruction number respectively.
See the \texttt{test/UnitTests/test4} for more information.

\subsection{The Design of Giri}
Giri handles both data-flow and control-flow dependences when computing the dynamic backwards slice. 
It has two phases.
In the first pass named \emph{tracing} pass, it instruments the code to record LLVM IR in a trace file at run-time for all threads,
In the second pass named \emph{slicing} pass, it uses the execution trace to create a program dependence graph for computing dynamic slice.
Giri takes advantage of the LLVM IR representation (static single assignment (SSA) form) to reduce the size of the trace file. 

The tracing pass instruments code to record three different pieces of information:
(a) basic block exits; (b) memory accesses and their addresses; and (c) function calls and returns.
When the slicing pass first adds instructions from a dynamic execution of a basic block to the backwards slice,
it uses static control-dependence analysis to determine which value forced the execution of that basic block.
Giri will then find the most recent execution of the instruction generating that value and add it to the backwards slice.
Since most instructions operate on SSA scalar values, we only refer to dynamic trace whenever needed.
For example the \texttt{load/store} address can be determined from the explicit SSA graph in the LLVM IR representation of the program.
The slicing pass ends when all dynamic values are processed and reports the dynamic slice of the specific program execution.

To make the slicing pass thread aware,
the runtime library traces the thread id (\texttt{thread\_t}) to each record indicating current thread performing that operation.
The slicing pass checks the thread id when scanning the trace file for a given dynamic value or basic block.

%------------------------------------------------
\section{Progress}
\label{sec:progress}
Our goal of GSoC was to make the Giri code up-to-date with the latest LLVM version, improve its runtime performance, and/or reduce the tracing overhead.
There are several things we did in this summer:
\begin{enumerate}
	\item \textbf{Update the code to LLVM mainline.}
		\begin{itemize}
			\item Release a v3.1 version which works with LLVM 3.1
			\item Make the code update to the latest LLVM 3.4 (\texttt{r191529})
		\end{itemize}
	\item \textbf{Make the Giri run-time library thread safe and the slicing pass thread aware}.
		The traces records of all threads are written to a single file.
		In addition to other information, trace records also include thread id (\texttt{pthread\_t}),
		which indicates the thread performing that particular operation.
		We use locks when when we write trace records to file in order to avoid race conditions.
	\item \textbf{Improve the Giri run-time performance.}
		We checked every call to function \texttt{mmap} as well as the parameters, and eliminated the useless ones.
		For example, at the end of the tracing, it is not needed to re-\texttt{mmap} once all traces are synchronized to disk file.
		Giri now dynamically computes the cache size of trace records to hold in memory before flushing to disk.
	\item \textbf{Write dozens of unit tests and try more real programs.}
		We had few large real program before like squid, apache, MySQL.
		There is also a simple test framework which runs all the unit tests at the top level of \texttt{giri/test/} and report the results.
		In the future development, every patch should make all unit tests pass before committing it to the git repository.
	\item \textbf{Reduce the trace size.}
		We truncate the trace file at the end of tracing according to its real size.
	\item \textbf{Write documents for the project.}\\
		There are several sample pages:
		\begin{itemize}
			\item \href{https://github.com/liuml07/giri/wiki}{The Wiki Page Home}.
			\item \href{https://github.com/liuml07/giri/wiki/How-to-Compile-Giri}{How to Compile Giri}.
			\item \href{https://github.com/liuml07/giri/wiki/Hello-World}{Hello World! Example}
			\item \href{https://github.com/liuml07/giri/wiki/Example-Usage}{Example Usage With An Example}.
		\end{itemize}
\end{enumerate}

%------------------------------------------------
\section{Future Work}
The following are the future work list we are going to address in the next few months.
An update list is maintained online at the \href{https://github.com/liuml07/giri/wiki/TODO}{wiki page of Giri at github.com}.
\label{sec:todo}
\begin{enumerate}
	\item Improve the performance of locking mechanism of the Giri runtime
	\item Write more unit tests with complicated direct/indirect recursive function calls
	\item Double check the \texttt{TraceFile.cpp} to make sure the thread id be checked correctly
	\item Add \texttt{tool/Tracer.cpp} to the make list. It was removed when upgrading to LLVM 3.4
	\item Try more real world programs and add them to the \texttt{test/} directory
	\item Try large test programs, e.g. \emph{nginx}, \emph{squid}, etc
	\item Make Giri code useful for other platforms besides Linux, e.g. Mac OS X, Cygwin, and FreeBSD.
	\item Consider more special function calls, which are intrinsic LLVM instructions (like \texttt{strcat} and \texttt{calloc}) in \texttt{TracingNoGiri::visitSpecialCall()}
	\item Parallelize the code which writes the entry cache containing trace records to trace file and adds trace records to the entry cache.
\end{enumerate}

%------------------------------------------------
\section{Conclusion}
\label{sec:contact}
As the first publicly available dynamic slicing tool,
Giri was made better during the GSoC. We updated the code to LLVM mainline, reduced the trace size, made it thread-aware, and improve the performance of its run-time.
We have published our code at \href{https://github.com/liuml07/giri}{https://github.com/liuml07/giri}~\cite{giri}.

The code is still under active development.
Dr. Swarup will direct Mingliang LIU in future to make the Giri code better.
There are other researchers from Tsinghua University, Xi'an Jiaotong University and University of Miami,
who contacted us in email hoping to use and contribute to Giri project.
For more information, please visit the homepage above.

%------------------------------------------------
\bibliographystyle{plain}
\bibliography{references} 

\end{document}
